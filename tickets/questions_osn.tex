osn 1. Предел и непрерывность функций одной и нескольких переменных. Свойства функций  непрерывных на отрезке.

osn 2. Производная и дифференциал функций одной и нескольких переменных. Достаточные условия дифференцируемости.

osn 3. Определенный интеграл, его свойства. Основная формула интегрального исчисления.

osn 4. Числовые ряды. Абсолютная и условная сходимость. Признаки сходимости: Даламбера, интегральный, Лейбница.

osn 5. Функциональные ряды. Равномерная сходимость. Признак Вейерштрасса. Непрерывность суммы  равномерно сходящегося ряда непрерывных функций.

osn 6. Криволинейный интеграл, формула Грина.

osn 7. Производная функции комплексногопеременного. Условия Коши-Римана. Аналитическая  функция.

osn 8. Степенные ряды в действительной и комплексной области. Радиус сходимости.

osn 9. Ряд Фурье по ортогональной системе функций. Неравенство Бесселя, равенство Парсеваля,  сходимость ряда Фурье.

osn 10. Прямая и плоскость, их уравнения. Взаимное расположение прямой и плоскости,  основные задачи на прямую и плоскость.

osn 11. Алгебраические линии и поверхности второго порядка, канонические уравнения,  классификация.

osn 12. Системы линейных алгебраических уравнений. Теорема Кронекера-Капелли. Общее решение системы линейных алгебраических уравнений.

osn 13. Линейный оператор в конечномерном пространстве, его матрица. Норма линейного оператора.

osn 14. Ортогональные преобразования евклидова пространства. Ортогональные матрицы и их свойства.

osn 15. Характеристический многочлен линейного оператора. Собственные числа и собственные векторы.

osn 16.Формализация понятия алгоритма. Машины Тьюринга, нормальные алгоритмы Маркова. Алгоритмическая неразрешимость. Задача останова. Задача самоприменимости.

osn 17. Понятие архитектуры ЭВМ. Принципы фон Неймана. Компоненты компьютера: процессор, оперативная память, внешние устройства. Аппарат прерываний.

osn 18. Операционные системы. Процессы, взаимодействие процессов, разделяемые ресурсы, синхронизация взаимодействующих процессов, взаимное исключение. Программирование взаимодействующих процессов с использованием средств ОС UNIX (сигналы, неименованные каналы, IPC).

osn 19. Системы программирования. Основные компоненты систем программирования, схема их функционирования. Общая схема работы компилятора. Основные методы, используемые при построении компиляторов.

osn 20. Основные принципы объектно-ориентированного программирования. Реализация этих принципов в языке С++. Примеры.

osn 21. Базы данных.Основные понятия реляционной модели данных. Реляционная алгебра. Средства языка запросов SQL.

osn 22. Виды параллельной обработки данных, их особенности. Компьютеры с общей и распределенной памятью. Производительность вычислительных систем, методы оценки и измерения.

osn 23. Основные методы обработки изображений: тональная коррекция, свёртка изображений, выделение краёв.

osn 24. Линейные обыкновенные дифференциальные уравнения и системы. Фундаментальная система решений. Определитель Вронского.

osn 25. Теоремы существования и единственности решения задачи Коши для обыкновенного дифференциального уравнения первого порядка, разрешенного относительно производной.

osn 26. Функции алгебры логики. Реализация их формулами. Совершенная дизъюнктивная нормальная форма.

osn 27. Схемы из функциональных элементов и простейшие алгоритмыих синтеза. Оценка сложности схем, получаемых по методу Шеннона.

osn 28. Вероятностное пространство. Случайные величины. Закон больших чисел в форме Чебышева.

osn 29. Квадратурные формулы прямоугольников, трапеций и парабол.

osn 30. Методы Ньютона и секущих для решения нелинейных уравнений.

osn 31. Численное решение задачи Коши для обыкновенных дифференциальных уравнений. Примеры методов Рунге-Кутта.

osn 32. Задача Коши для уравнения колебания струны. Формула Даламбера.

osn 33. Постановка краевых задач для уравнения теплопроводности.  Метод разделения переменных для решения первой краевой задачи.