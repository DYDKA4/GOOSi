\subsection{DOP 8 Зависимости  в  реляционных  отношениях:  функциональные,  многозначные,  проекции/соединения. Проектирование реляционных БД на основе принципов нормализации отношений. Нормальные формы.}

\textbf{Зависимости в реляционных отношениях}

\mathLet \ задана переменная отношения $R$ (=таблица), и $X$ и $Y$ являются произвольными подмножествами заголовка $R$ (<составными> атрибутами, наборами столбцов).

Атрибут $Y$ \textbf{функционально зависит} от атрибута $X \iff$ каждому значению $X$ соответствует в точности одно значение $Y$.
В этом случае говорят также, что атрибут $X$ \textbf{функционально определяет} атрибут $Y$ ($X$ является детерминантом (определителем) для $Y$, а $Y$ является зависимым от X). (далее, FD (Functional Dependency) --- функциональная зависимость).

FD $A \Rightarrow B$ --- \textbf{тривиальная}, если $A \supseteq B$. 
Любая тривиальная FD всегда выполнима.

Армстронг предложил правила вывода новых FD на основе существующих.
Аксиомы Армстронга:
\begin{enumerate}
    \item Если $A \supseteq B$ , то $A \Rightarrow B$ (\textbf{рефлексивность}).
    \item Если $A \Rightarrow B$, то $(A~UNION~C) \Rightarrow (B~UNION~C)$ (\textbf{пополнение}).
    \item Если $A \Rightarrow B$ и $B \Rightarrow C$, то $A \Rightarrow C$ (\textbf{транзитивность}).
\end{enumerate}


В переменной отношения $R$ с атрибутами $A$, $B$, $C$ (в общем случае, составными) имеется \textbf{многозначная зависимость} $B$ от $A$ ($A \Rightarrow\Rightarrow B$) $\iff$ множество значений атрибута $B$, соответствующее паре значений атрибутов $A$ и $C$, зависит от значения $A$ и не зависит от значения $C$.

\textbf{Декомпозицией} отношения $R$ называется замена $R$ на совокупность отношений $\{R_1, R_2, \dots , R_n\}$ такую, что каждое из них есть проекция $R$, и каждый атрибут $R$ входит хотя бы в одну из проекций декомпозиции.
То есть все $R_i$ состоят только из атрибутов $R$ и любой атрибут $R$ есть хотя бы в одном $R_i$.

$\mathLet ~ R' = NATURAL~JOIN~(R_1, \dots, R_n)$.
Декомпозиция $\{R_1, R_2, \dots , R_n\}$ называется \textbf{декомпозицией без потерь}, если $R' = R$

$\mathLet  ~ \exists$ некоторое отношение $r$ со схемой $R$, а также два произвольных подмножества атрибутов $A,B\subseteq R$. 
$\mathLet ~ C = R \setminus (A\cup B)$.
В этом случае $B$ \textbf{многозначно зависит} от $A$, тогда и только тогда, когда множество значений атрибута $B$, соответствующее заданной паре $[a:A;c:C]$ отношения $r$, зависит от $a$ и не зависит от $c$.

$\mathLet ~ R$ --- переменная отношения, а $A, B, \dots, Z$ --- некоторые подмножества множества её атрибутов.
Если декомпозиция любого допустимого значения $R$ на отношения, состоящие из множеств атрибутов $A, B, \dots, Z$, является декомпозицией без потерь, говорят, что переменная отношения $R$ удовлетворяет зависимости соединения $\ast\{A, B, \dots, Z\}$.

\textbf{Проектирование реляционных БД}

\textbf{Теорема Хита.} 
\mathLet \ Пусть $r \{A, B, C\}$ - отношение, состоящее из множества атрибутов $A$, $B$ и $C$. Если в R есть функциональная зависимость $A \Rightarrow B$, то R равно соединению его проекций \{A, B\} \{A, С\}.


\textbf{Нормальные формы}

Переменная отношения \textbf{находится в 1НФ} тогда и только тогда, когда в любом допустимом значении отношения каждый его кортеж содержит только одно значение для каждого из атрибутов. 
Реляционное отношение уже находится в 1НФ.

\textbf{Замыкание множества} FD $S$ --- множество FD $S^+$, включающее все FD, логически выводимые из множества $S$.

FD с минимальным детерминантом (удаление любого атрибута из детерминанта приводит к изменению замыкания $S^+$ , т. е. порождению множества FD, неэквивалентного $S$) называется \textbf{минимальной слева}. 

Атрибут B \textbf{минимально зависит} от атрибута A, если выполняется минимальная слева FD $A \Rightarrow B$.

\textbf{Потенциальный ключ} --- в реляционной модели данных --- подмножество атрибутов отношения, удовлетворяющее требованиям уникальности и минимальности (несократимости).
\begin{itemize}
    \item Уникальность означает, что нет и не может быть двух кортежей данного отношения, в которых значения этого подмножества атрибутов совпадают (равны).
    
    Свойство уникальности определяется не для конкретного значения переменной отношения в тот или иной момент времени, а по всем возможным значениям, то есть следует из внешнего знания о природе и закономерностях данных, которые могут находиться в переменной отношения.
    \item Минимальность (несократимость) означает, что в составе потенциального ключа отсутствует меньшее подмножество атрибутов, удовлетворяющее условию уникальности.
    Иными словами, если из потенциального ключа убрать любой атрибут, он утратит свойство уникальности.
\end{itemize}

\textbf{Первичный ключ} --- в реляционной модели данных один из потенциальных ключей отношения, выбранный в качестве основного ключа (или ключа по умолчанию).

\textbf{Суперключ отношения $r$} --- любое подмножество $K$ заголовка $r$, включающее, по меньшей мере, хотя бы один возможный ключ $r$.

FD $A \Rightarrow C$ называется \textbf{транзитивной}, если существует такой атрибут $B$, что имеются ФЗ и $A \Rightarrow B$ и $B \Rightarrow C$ и отсутствует ФЗ $C \Rightarrow A$.

Переменная отношения \textbf{находится в 2НФ} тогда и только тогда, когда она находится в 1НФ, и каждый неключевой атрибут минимально функционально зависит от первичного ключа.

Данные находятся в 1 НФ но не 2 НФ:

\begin{tabular}{|c|c|c|c|}
    \hline
    \textbf{PK: Модель} & \textbf{PK: Фирма} & Цена & Скидка \\
    \hline
    \hline
    M5  & BMW     & 50k\$ & 5\% \\
    \hline
    X5M & BMW     & 51k\$ & 5\% \\
    \hline
    GT-R & Nissan & 47k\$ & 10\% \\
    \hline
\end{tabular}

Цена машины зависит от модели и фирмы. Скидка зависят от фирмы, то есть зависимость от первичного ключа неполная. Исправляется это путем декомпозиции на два отношения, в которых не ключевые атрибуты зависят от ПК.

\begin{tabular}{cc}
    \begin{minipage}{.55\linewidth}
    
        \begin{tabular}{|c|c|c|}
            \hline
            \textbf{PK: Модель} & \textbf{PK: Фирма} & Цена \\
            \hline
            \hline
            M5  & BMW     & 50k\$ \\
            \hline
            X5M & BMW     & 51k\$ \\
            \hline
            GT-R & Nissan & 47k\$ \\
            \hline
        \end{tabular}

    \end{minipage} &
    \begin{minipage}{.55\linewidth}
    
        \begin{tabular}{|c|c|}
            \hline
            \textbf{PK: Фирма} & Скидка \\
            \hline
            \hline
            BMW     & 5\% \\
            \hline
            Nissan  & 10\% \\
            \hline
        \end{tabular}
        
    \end{minipage} 
\end{tabular}

\bigbreak

Переменная отношения \textbf{находится в 3НФ} тогда и только тогда, когда она находится в 2НФ и каждый неключевой атрибут нетранзитивно функционально зависит от первичного ключа. 

Данные находятся в 2 НФ но не 3 НФ:

\begin{tabular}{|c|c|c|}
    \hline
    \textbf{PK: Модель} & Магазин & Телефон \\
    \hline
    \hline
    BMW    & Salon Lan-bin & 18-05-00 \\
    \hline
    Audi   & Salon Lan-bin & 18-05-00 \\
    \hline
    Nissan & Cherep-avto   & 07-04-99 \\
    \hline
\end{tabular}

В отношении атрибут «Модель» является первичным ключом. Личных телефонов у автомобилей нет, и телефон зависит исключительно от магазина.
Таким образом, в отношении существуют следующие функциональные зависимости: Модель $\rightarrow$ Магазин, Магазин $\rightarrow$ Телефон, Модель $\rightarrow$ Телефон.
Зависимость Модель $\rightarrow$  Телефон является транзитивной, следовательно, отношение не находится в 3НФ.
В результате разделения исходного отношения получаются два отношения, находящиеся в 3НФ:

\begin{tabular}{cc}
    \begin{minipage}{.5\linewidth}
    
        \begin{tabular}{|c|c|}
            \hline
            \textbf{PK: Модель} & Магазин \\
            \hline
            \hline
            BMW    & Salon Lan-bin \\
            \hline
            Audi   & Salon Lan-bin \\
            \hline
            Nissan & Cherep-avto   \\
            \hline
        \end{tabular}

    \end{minipage} &
    \begin{minipage}{.5\linewidth}
    
        \begin{tabular}{|c|c|}
            \hline
            \textbf{PK: Магазин} & Телефон \\
            \hline
            \hline
            Salon Lan-bin & 18-05-00 \\
            \hline
            Cherep-avto   & 07-04-99 \\
            \hline
        \end{tabular}
        
    \end{minipage} 
\end{tabular}

\bigbreak

\textbf{BCNF, 4НФ}
Между 3 и 4 нормальной формой есть еще и промежуточная нормальная форма, она называется – \textbf{Нормальная форма Бойса-Кодда (BCNF)}. Иногда ее еще называют «Усиленная третья нормальная форма».
\begin{enumerate}
    \item Таблица должна находиться в третьей нормальной форме. Здесь все как обычно, т.е. как и у всех остальных нормальных форм, первое требование заключается в том, чтобы таблица находилась в предыдущей нормальной форме, в данном случае в третьей нормальной форме
    \item Ключевые атрибуты составного ключа не должны зависеть от неключевых атрибутов
\end{enumerate}

Отсюда следует, что требования нормальной формы Бойса-Кодда предъявляются только к таблицам, у которых первичный ключ составной. Таблицы, у которых первичный ключ простой, и они находятся в третьей нормальной форме, автоматически находятся и в нормальной форме Бойса-Кодда.

Требование \textbf{четвертой нормальной формы (4NF)} заключается в том, чтобы в таблицах отсутствовали нетривиальные многозначные зависимости и была в 3NF.
В таблицах многозначная зависимость выглядит следующим образом:
\textbf{Таблица должна иметь как минимум три столбца, допустим A, B и C, при этом B и C между собой никак не связаны и не зависят друг от друга, но по отдельности зависят от A, и для каждого значения A есть множество значений B, а также множество значений C.}

\textbf{В данном случае многозначная зависимость обозначается вот так: $A \Rightarrow B, A \Rightarrow C$}
Если подобная многозначная зависимость есть в таблице, то она \textbf{не соответствует} четвертой нормальной форме.

% Отношения $a$, $b$ могут обновляться независимо, если являются \textit{независимыми проекциями}.

% Необходимые и достаточные условия независимости проекций отношения обеспечивает теорема Риссанена: Проекции $r_1$ и $r_2$ отношения $r$ являются независимыми тогда и только тогда, когда: 
% \begin{itemize}
%     \item Каждая FD в отношении $r$ выводима на основании аксиом Армстронга из FD в $r_1$ и $r_2$.
%     \item Общие атрибуты $r_1$ и $r_2$ образуют возможный ключ хотя бы для одного из этих отношений.
% \end{itemize}

% Нормальная форма Бойса-Кодда --- уточнение 3НФ в случае наличия нескольких перекрывающихся возможных ключей.
% Переменная отношения \textit{находится в BCNF} тогда и только тогда, когда любая выполняемая для этой переменной отношения нетривиальная и минимальная FD имеет в качестве детерминанта некоторый возможный ключ данного отношения. 

% \textit{Лемма Фейджина.} 
% В отношении $R~A,~B,~C$ выполняется MVD $A \Rightarrow\Rightarrow B$ в том и только в том случае, когда выполняется MVD $A \Rightarrow\Rightarrow C$. 

% Таким образом, MVD $A \Rightarrow\Rightarrow B$ и $A \Rightarrow\Rightarrow C$ всегда составляют пару. 
% Поэтому обычно их представляют вместе в форме $A \Rightarrow\Rightarrow B | C$. 

% \textit{Теорема Фейджина.} 
% Пусть имеется переменная отношения $R$ с атрибутами $A$, $B$, $C$ (в общем случае, составными). 
% Отношение $R$ декомпозируется без потерь на проекции $\{A, B\}$ и $\{A, C\}$ тогда и только тогда, когда для него выполняется MVD $A \Rightarrow\Rightarrow B | C$.

% Переменная отношения $r$ \textit{находится в 4НФ} тогда и только тогда, когда она находится в BCNF, и все MVD $r$ являются FD с детерминантами --- возможными ключами отношения $r$. 
% В сущности, 4НФ является BCNF, в которой многозначные зависимости вырождаются в функциональные.

% Отношение \textit{находится в 5НФ} тогда и только тогда, когда каждая нетривиальная зависимость соединения в нём определяется потенциальным ключом (ключами) этого отношения.

% Зависимость соединения $\ast\{A, B, \dots, Z\}$ определяется потенциальным ключом (ключами) тогда и только тогда, когда каждое из подмножеств $A, B, \dots, Z$ множества атрибутов является суперключом отношения.

% -------- source --------
\bigbreak
[\cite{bd_type}]