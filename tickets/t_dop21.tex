% -+- используется в dop 21
\newcommand{\notsure}[1]{(видимо: #1)}
\newcommand{\aboba}[2]{\textbf{\LARGE dop #1. #2}}
\newcommand{\multicom}[1]{}
\newcommand{\lulz}[1]{}
\newcommand{\wantsayInstead}[1]{}
% -+- 

\subsection{DOP 20 Синхронизация  в  распределенных  системах.  Синхронизация  времени.  Логические  часы.  Выборы координатора. Взаимное исключение. Координация процессов.}

\bigbreak

\centerline{\textbf{Основное}}

\textit{Распределенная (компьютерная) система} (РС) --– совокупность связанных
сетью независимых компьютеров, которая представляется
пользователю единым компьютером. 

% Черты РС:
% \begin{itemize}
%     \item конкурентность \notsure{доступ к ресурсам};
%     \item отсутствие глобальных часов;
%     \item независимые отказы.
% \end{itemize}

Свойства децентрализованных алгоритмов:
\vspace{-0.7em}
\begin{enumerate}
\setlength\itemsep{-0.4em}
\item используемая информация распределена среди множества ЭВМ;
\item процессы действуют на основе только локальной информации;
\item отсутствие критического узла, выход из строя которого приводит к краху алгоритма;
% eq:up \item не должно быть единственной критической точки, выход из строя которой приводил бы к краху алгоритма;
\item отсутствие общего источника глобального времени.
\end{enumerate}
пункты 1--3 о недопустимости хранения всей информации необходимой для принятия решения в одном место.


\centerline{\textbf{Синхронизация времени}}
%\begin{itemize}

$\circ$
\textbf{Аппаратные часы} --- счетчик временных сигналов и регистр с начальным значением счетчика [из слайдов].
\\
(\textit{Аппаратные часы} --- счетчик времени, система содержащая автономный источник питания и регистр. [из вики]).
%\wantsayInstead{\textit{Аппаратные часы} --- счетчик времени, система из автономного источника питания и регистра.}

$\circ$
Отношение ``произошло до'' ($\rightarrow$): $a \rightarrow b$ означает, что все процессы согласны, что сначала произошло событие $a$, а затем $b$. 
Оно очевидно в 2--х случаях:
\vspace{-0.7em}
\begin{itemize}
    \setlength\itemsep{-0.4em}
    \item \lulz{аб}оба события ($a$ и $b$) произошли в одном процессе;
    \item событие $a$ --- отправка сообщения $m$, событие $b$ - прием $m$.
\end{itemize}

Отношение $\rightarrow$ транзитивно ($a \rightarrow b$ и $b \rightarrow c$ тогда $a \rightarrow c$).

Если события $x$ и $y$ произошли в разных процессах, не обменивающихся сообщениями, то отношения $x \rightarrow y$ и $y \rightarrow x$ неверны. 
Такие события $x$ и $y$ называются \textit{одновременными}.

$\circ$
\textbf{\textit{Логические часы}} --- согласованное (между ЭВМ РС) время (потенциально) не имеющее ничего общего с астрономическим временем.

Введем логическое время $C$ следующим образом: если $a \rightarrow b$, то $C(a) < C(b)$.
Логические часы используют следующий алгоритм:
\begin{enumerate}
\item 
Часы $C_i$ увеличивают свое значение с каждым событием в процессе $P_i$:
$C_i = C_i + d$, где $d > 0$ (обычно $1$).

\item 
Если событие $a$ --- отправка сообщения $m$ процессом $P_i$, то в него добавляется временная метка $t_m=C_i(a)$.
При получении сообщения $m$ процессом $P_j$ его время корректируется: \\ $C_j = max(C_j,t_m + d$).
\end{enumerate}
%\end{itemize}

\centerline{\textbf{Выбор координатора}}

Многие распределенные алгоритмы требуют, чтобы один из процессов выполнял функции координатора. Рассмотрим алгоритмы выбора координатора.
\begin{enumerate}
\item 
Алгоритм ``Задира'':
\begin{itemize}
    \setlength\itemsep{-0.4em}
    \item Если процесс $P$ обнаружит, что координатор долго не отвечает, то он инициирует выборы.
    \item $P$ посылает сообщение ``ВЫБОРЫ'' всем процессам с большими чем у него номерами.
    \item Если нет ни одного ответа, то $P$ считается победителем и становится координатором.
    \item Если один из процессов с большим номером ответит, то он берет на себя проведение выборов. Участие процесса $P$ в выборах заканчивается.
    \item В любой момент процесс может получить сообщение ``ВЫБОРЫ'' (от процесса с меньшим номером). В этом случае он посылает ответ ``OK'', чтобы сообщить, что он жив и берет проведение выборов на себя.
    \item Победитель извещает всех о своей победе сообщением ``КООРДИНАТОР''.
    \item (Сомнительный пункт: типа зачем, если новый координатор умрет, то управление вернется) Если бывший координатор восстановился после сбоя, то он проводит выборы.
\end{itemize}
 
\item 
Круговой алгоритм:
\begin{itemize}
\item
Каждый процесс знает следующего за ним в круговом списке.
\item
Когда процесс обнаруживает отсутствие координатора, он посылает следующему за ним процессу сообщение ``ВЫБОРЫ'' со своим номером.
\item
Если следующий процесс не отвечает, то сообщение посылается процессу, следующему за ним, и т.д., пока не найдется работающий процесс.
\item
Каждый работающий процесс добавляет в список работающих свой номер и переправляет сообщение дальше по кругу. 
\item
Когда процесс обнаружит в списке свой собственный номер (круг пройден), он меняет тип сообщения на ``КООРДИНАТОР'' и оно проходит по кругу, извещая всех о списке работающих и координаторе.
\end{itemize}
\end{enumerate}


\centerline{\textbf{Взаимное исключение}}

Речь о доступе к ресурсам. Рассмотрим несколько алгоритмов:
\begin{enumerate}
\item 
Централизованный алгоритм.
\begin{itemize}
\item 
Все процессы запрашивают у координатора разрешение на вход в критическую секцию и ждут этого разрешения. 
\item 
Координатор обслуживает запросы в порядке поступления. 
\item 
Получив разрешение процесс входит в критическую секцию. 
\item 
При выходе из нее он сообщает об этом координатору. 
\end{itemize}
Кол--во сообщ. на одно прохождение критической секции --- 3.
\\
Недостатки алгоритма --- крах координатора или его перегрузка сообщениями.

\item 
Алгоритм с круговым маркером.
\begin{itemize}
\item Каждый процесс знает следующего за ним в круговом списке.
\item По кольцу циркулирует маркер, дающий право на вход в критическую секцию. 
\item Получив маркер (специальное сообщение) процесс либо входит в критическую секцию, либо переправляет маркер дальше.
\item После выхода из критической секции маркер переправляется дальше.
\end{itemize}

\item
Децентрализованный алгоритм на основе временных меток.
\\
Необх. глобальное упорядочение всех событий в сист. по времени.

Вход в критическую секцию:
\begin{itemize}
\item
Когда процесс желает войти в критическую секцию, он посылает всем процессам сообщение-запрос, содержащее имя критической секции, номер процесса и текущее время.
\item После посылки запроса процесс ждет, пока все дадут ему разрешение.
\item После получения от всех разрешения, он входит в критическую секцию.
\end{itemize}

Поведение процесса при приеме запроса:
\begin{itemize}
\item
Если получатель не находится внутри критической секции и не запрашивал разрешение на вход в нее, то он посылает отправителю сообщение``OK''.
\item
Если получатель находится внутри критической секции, то он не отвечает, а запоминает запрос.
\item
Если получатель выдал запрос на вхождение в эту секцию, но еще не вошел в нее, то он сравнивает временные метки своего запроса и чужого.
Побеждает тот, чья метка меньше. 
Если чужой запрос победил, то процесс посылает сообщение ``OK''. 
Если у чужого запроса метка больше, то ответ не посылается, а чужой запрос запоминается.
\end{itemize}

Выход из критической секции:
\begin{itemize}
\item
После выхода из секции он посылает сообщение «OK» всем процессам, запросы от которых он запомнил, а затем стирает все запомненные запросы.
\end{itemize}

Количество сообщений на одно прохождение секции --- $2(n-1)$, где n --- число процессов.
%\\
Если какой-то процесс перестанет функционировать, то отсутствие разрешения от него всех остановит.
% \\
Если в централизованном алгоритме есть опасность перегрузки координатора, то в этом алгоритме перегрузка любого процесса приведет к тем же последствиям.


%not uncom
%\item И еще несколько алгоритмов смотри ниже (писать не предлагаю), под чертой, в секции Cringe пункты 1--3. 

\end{enumerate}

\centerline{\textbf{Координация процессов}}
\begin{enumerate}
\item  Если известен и ``потребитель'' и ``производитель'': 
\begin{itemize} \item cообщения точка--точка. \end{itemize}
\item Если \textbf{НЕ}известен ``потребитель'':
\begin{itemize}
    \item широковещательные сообщения;
    \item сообщения в ответ на запрос.
\end{itemize}
\item Если \textbf{НЕ}известен и ``потребитель'' и ``производитель'' :
\begin{itemize}
    \item сообщения и запросы через координатора:
    \item широковещательный запрос.
\end{itemize}
\end{enumerate}

%\hrulefill

%\textbf{Cringe}
% У вас не захотят это спрашивать, вы не захотите это рассказывать:
% %\begin{enumerate}
% \\$\circ$%\item 
% Децентрализованный алгоритм на основе временных меток.
% \\
% Требуется глобальное упорядочение всех событий в системе по времени.
% \begin{enumerate}

% \item 
% Вход в критическую секцию:
% \begin{itemize}
% \item
% Когда процесс желает войти в критическую секцию, он посылает всем процессам сообщение-запрос, содержащее имя критической секции, номер процесса и текущее время.
% \item После посылки запроса процесс ждет, пока все дадут ему разрешение.
% \item После получения от всех разрешения, он входит в критическую секцию.
% \end{itemize}

% \item
% Поведение процесса при приеме запроса:
% \begin{itemize}
% \item
% Если получатель не находится внутри критической секции и не запрашивал разрешение на вход в нее, то он посылает отправителю сообщение``OK''.
% \item
% Если получатель находится внутри критической секции, то он не отвечает, а запоминает запрос.
% \item
% Если получатель выдал запрос на вхождение в эту секцию, но еще не вошел в нее, то он сравнивает временные метки своего запроса и чужого.
% Побеждает тот, чья метка меньше. 
% Если чужой запрос победил, то процесс посылает сообщение ``OK''. 
% Если у чужого запроса метка больше, то ответ не посылается, а чужой запрос запоминается.
% \end{itemize}

% \item
% Выход из критической секции:
% \begin{itemize}
% \item
% После выхода из секции он посылает сообщение «OK» всем процессам, запросы от которых он запомнил, а затем стирает все запомненные запросы.
% \end{itemize}
% \end{enumerate}
% Количество сообщений на одно прохождение секции - 2(n-1), где n - число процессов.
% \\
% Если какой-то процесс перестанет функционировать, то отсутствие разрешения от него всех остановит.
% \\
% И, наконец, если в централизованном алгоритме есть опасность перегрузки координатора, то в этом алгоритме перегрузка любого процесса приведет к тем же последствиям.



% \item
% Алгоритм широковещательный маркерный (прям уверен что это писать не стоит):
% \\
% Маркер содержит: 
% \begin{itemize}
% \item очередь запросов;
% \item массив $LN[1...N]$ с номерами последних удовлетворенных запросов.
% \end{itemize}
% Сам Алгоритм:
% \begin{itemize}
% \item
% Вход в критическую секцию
% \begin{itemize}
% \item
% Если процесс $P_k$, запрашивающий критическую секцию, не имеет маркера, то он увеличивает порядковый номер своих запросов $RN_k[k]$ (я не ебу кто такой $RN_k$) и посылает широковещательно сообщение ``ЗАПРОС'', содержащее номер процесса $k$ и номер запроса $S_n = RN_k[k]$.
% \item
% Процесс $P_k$ выполняет критическую секцию, если имеет (или когда получит) маркер.
% \end{itemize}
% \item
% Поведение процесса при приеме запроса
% \begin{itemize}
% \item 
% Когда процесс $P_j$ получит сообщение-запрос от процесса $P_k$, он
% устанавливает $RN_j[k]=max(RN_j[k],S_n)$. Если Pj имеет свободный маркер,
% то он его посылает $P_k$ только в том случае, когда $RN_j[k]==LN[k]+1$
% (запрос не старый).
% \end{itemize}
% \item
% Выход из критической секции процесса $P_k$.
% \begin{itemize}
% \item Устанавливает $LN[k]$ в маркере равным $RN_k[k]$.
% \item Для каждого $P_j$, для которого $RN_k[j]=LN[j]+1$, он добавляет его идентификатор в маркерную очередь запросов (если там его еще нет).
% \item Если маркерная очередь запросов не пуста, то из нее удаляется первый элемент, а маркер посылается соответствующему процессу (запрос которого был первым в очереди).
% \end{itemize}
% \end{itemize}

% \item
% Алгоритм древовидный маркерный.
% \\
% Все процессы представлены в виде сбалансированного двоичного дерева.
% Каждый процесс имеет очередь запросов от себя и соседних процессов (1-ого,
% 2-х или 3-х) и указатель в направлении владельца маркера.
% \\
% Критическая секция --- КС.
% \\
% Сам алгоритм:
% \begin{itemize}
% \item Вход в критическую секцию
% \begin{itemize}
%     \item Если есть маркер, то процесс выполняет КС.
%     \item Если нет маркера, то процесс:
% \begin{itemize}
%     \item помещает свой запрос в очередь запросов.
%     \item посылает сообщение «ЗАПРОС» в направлении владельца маркера и ждет сообщений.
% \end{itemize} 
% \end{itemize}
% \item 
% Поведение процесса при приеме сообщений.
% \\
% Процесс, не находящийся внутри КС должен реагировать на сообщения двух
% видов ``МАРКЕР'' и ``ЗАПРОС''.
% \begin{itemize}
%     \item 
% Пришло сообщение ``МАРКЕР'':
% \begin{itemize}
% \item пункт $M1$. 
% Взять 1-ый запрос из очереди и послать маркер его автору (концептуально, возможно себе);
% \item  Поменять значение указателя в сторону маркера;
% \item  Исключить запрос из очереди;
% \item  Если в очереди остались запросы, то послать сообщение ``ЗАПРОС'' в сторону маркера.
% \end{itemize}
% \item 
% Пришло сообщение ``ЗАПРОС''.
% \begin{itemize}
%     \item  Поместить запрос в очередь.
%     \item  Если нет маркера, то послать сообщение ``ЗАПРОС'' в сторону маркера,
% иначе (если есть маркер) - перейти на пункт М1.
% \end{itemize} 
% \end{itemize}
% \end{itemize}

% \item
% Выход из критической секции:
% \\
% Если очередь запросов пуста, то при выходе ничего не делается, иначе - перейти к пункту М1.
% \end{enumerate}


% -------- source --------
\bigbreak
[\cite[КонспектЛекций.zip/3-4-MPI-Sync.doc: page 8--15]{keldysh_ros_2021}]
