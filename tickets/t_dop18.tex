\subsection{DOP 17 Базисные типы данных в языках программирования. Основные проблемы, связанные с базисными типами и способы их решения в различных языках. Понятие абстрактного типа данных и способы его реализации в современных языках программирования.}

\textbf{Простые типы данных и их свойства}

\textbf{Целые типы:}
\begin{itemize}
    \item Универсальность (насколько полно учтены машинные типы).
    \item Наличие (или отсутствие) беззнаковых типов (в Java их нет, в С++, С\# есть).
    \item Представление (размер значения, диапазоны значений).
    \item Надежность (какие ошибки могут возникать при выполнении операций с целыми значениями --- переполнение типа)
    \item Набор операций (почти во всех языках одинаково, в Java нет беззнакового типа, поэтому есть логический сдвиг).
\end{itemize}

\textbf{Вещественные типы:}
\begin{itemize}
    \item $(-1)^S \ast M \ast 2^p$, где $S$ --- бит знака, $M$ --- нормализованная мантисса ($0.5 \leq M < 1$), $p$ --- порядок.
    \item Неточные (например, \texttt{float} --- точность сложения $2^{-23}$ , если операнды между 0.5 и 1.
\end{itemize}

\textbf{Символьные типы:}
\begin{itemize}
    \item Включают в себя как символы алфавитов естественных языков, так и символы, управляющие работой устройств ввода/вывода, и специальные символы.
    \item Главной проблемой символьного типа является выбор кодировки.
    Современное решение --- Unicode.
\end{itemize}

\textbf{Логические типы:} в С отсутствует, в С++ можно преобразовывать к целому, в C\#, Java --- нет.

\textbf{Порядковые типы:}
\begin{itemize}
    \item \textit{Перечислимые типы} --- перечень именованных значений констант, \textit{типы диапазона}. 
    \item \textit{Перечислимый тип С++}: числовые значения констант - всегда \texttt{int}, преобразования \texttt{enum} в \texttt{int} --- неявно, \texttt{int} в \texttt{enum} --- только явно, константы перечислимого типа имеют ту же область действия, что и имя перечислимого типа.
    \item \textit{Перечислимый тип C\#}: константы типа доступны через <имя типа>.<имя константы>, только явные преобразования между \texttt{enum} и \texttt{int}.
    \item \textit{Java}: аналогично C\# и являются классами.
    \item \textit{Тип диапазона} позволяет ограничить значения целого типа.
    Есть в Паскале, в С++, C\#, Java --- нет.
\end{itemize}

\textbf{Указательные и ссылочные типы данных:}
\begin{itemize}
    \item \textit{Указатель} абстракция понятия машинного адреса, есть в C, C++ --- может привести к хитрым ошибкам.
    Основные причины использования указателей: передача адресов объектов данных в подпрограммы, работа с объектами из динамической памяти.
    \item \textit{Ссылка} - абстракция понятия машинного адреса, лишенная недостатков указателя. 
    Ссылки C\# и Java отличаются от ссылок С++ тем, что ссылка С++ ``навсегда'', то есть в течение всего времени жизни ссылки, полностью ассоциированы с объектом. 
    Однако, и ссылки в С++, и ссылки в C\# и Java похожи в том смысле, что после установления ассоциации с объектом ссылка идентична самому объекту, поэтому не требуется никакая операция разыменования.
\end{itemize}

\textbf{Составные типы и их свойства}

\begin{enumerate}
    \item \textit{Одномерные массивы.}
    
    Массив — это непрерывная последовательность элементов одного типа.
    Атрибуты массива: базовый тип (T), тип индекса (I), диапазон индекса (L и R — нижняя и верхняя граница) и связанная с ним характеристика: длина массива. 
    В С\# и Java массивы - полноправные объекты.
    \item \textit{Многомерные массивы.}
    
    В большинстве языков рассматриваются как массивы массивов.
    В Java все многомерные массивы --- ступенчатые (то есть внутренние массивы не обязаны иметь одну длину), в C\# есть прямоугольный массив (для более эффективного доступа к элементам и возможности обрабатывать массивы, совместимые с моделью данных языков типа С).
    \item \textit{Динамические строки.}
    
    Последовательность символов произвольной длины.
    Необходимость введения специального типа вместо массива: строки реализуются как неизменяемый объект (главный аргумент), набор операций для строк существенно шире и специфичней набора операций для обычных массивов.
    \item \textit{Записи (структуры)} --- это совокупность объявлений переменных, которые объединены в отдельный объект.
\end{enumerate}

\textbf{Абстрактные типы данных}

\textit{Абстрактный тип данных (АТД)} --- тип, в котором внутренняя структура данных полностью инкапсулирована, то есть тип представлен только множеством операций.
Класс является абстрактным типом данных, если открытыми членами являются только методы. 

Говорят, что совокупность открытых членов класса составляет \textit{интерфейс} класса. 

\textit{Реализация:}
\begin{itemize}
    \item Сокрытие членов-данных, доступ через селекторы (геттеры-сеттеры) или их абстракцию --- свойство (в C\#).
    \item \textit{Абстрактный класс} --- класс, который предназначен исключительно для того, чтобы быть базовым классом.
    \item \textit{Интерфейс} --- класс, состоящий только из абстрактных методов.
\end{itemize}

% -------- source --------
\bigbreak
[\cite{golovin}]